%%%%%%%%%%%%%%% Start of Title Page %%%%%%%%%%%
\begin{titlepage}
  \begin{center}
    \textit{Heaven's Light is Our Guide}
    \\[0.5cm]
    \textbf{\Large Rajshahi University of Engineering \& Technology}
    \\[0.3cm]
    \textbf{\large Department of Electronics \& Telecommunication Engineering}
    \\[0.2cm]
    \begin{figure}[!htbp]
      \centering
      \includegraphics[scale=0.3]{Figures/logo_ruet}
      \label{fig:RUET logo}
    \end{figure}
    \textbf{\Large EEE1254: Sessional Based on EEE1253 }
    \\[0.5cm]
    \myrule[1pt][5pt]

    %%%%%%%%%%%%%%%%%%%%%%%%%%%%%%%%%%%%%%%%%%%%%%%
    %%%%%%%%%%%%%%% STUDENT'S INFO %%%%%%%%%%%%%%%%
    %%%%%%%%%%%%%%%%%%%%%%%%%%%%%%%%%%%%%%%%%%%%%%%

    \textbf{\Large  Experiment No. 01}
    \\[.25cm]
    \textbf{\large Observation of Ratings of Various Electrical Machines.}
    \\
    \myrule[1pt][5pt]
    \begin{minipage}{0.4\textwidth}
      \vspace{0.5cm}
      \begin{flushleft}
        \emph{\textbf{\large Submitted by:}}
        \\
        Shahoriar Rahman \\
        Roll: 2104001 \\
        Session: 2021-22
      \end{flushleft}
    \end{minipage}
    ~
    \begin{minipage}{0.4\textwidth}
      \vspace{0.5cm}
      \begin{flushright}
        \emph{\textbf{\large Submitted to:}}
        \\
        A. S. M. Badrudduza
        \\
        Assistant Professor
        \\
        Dept. of ETE, RUET
        \\
      \end{flushright}
    \end{minipage}\\[0.7cm]
    \makeatother

    \textbf{Date of Experiment : 08/07/2023}\\
    \textbf{Date of Submission : 22/07/2023}\\[1cm]

    %************** End of Student's Info *********


    \vfill
    %%%%%%%%%%%%%%% TEACHER SECTION %%%%%%%%%%%%%%%
    \hrulefill
    \vspace{-5mm}
    \begin{multicols}{3}
      \begin{itemize} [labelindent=3em,labelsep=0.5cm,leftmargin=*,noitemsep]
        \item[] \textbf{\underline{Report}}
        \item[$\square$] Excellent
        \item[$\square$] Very Good
        \item[$\square$] Good
        \item[$\square$] Average
        \item[$\square$] Poor
      \end{itemize}
      \columnbreak
      \textbf{(Teacher's Section)}
      \\[1.5cm]
      --------------------------------
      \\
      Signature
      \columnbreak
      \begin{itemize}
        [labelindent=6em,labelsep=0.5cm,leftmargin=*,noitemsep]
        \item[] \textbf{\underline{Viva}}
        \item[$\square$] Excellent
        \item[$\square$] Very Good
        \item[$\square$] Good
        \item[$\square$] Average
        \item[$\square$] Poor
      \end{itemize}
    \end{multicols}
  \end{center}
\end{titlepage}
\newpage
%************** End of Teacher Section ********
%************** End of Title Page *************


%%%%%%%%%%%%%%% Exp. Positioning %%%%%%%%%%%%%%
\titleformat{\chapter}[display]
{\normalfont\large\bfseries}{\chaptertitlename\ \thechapter}{0pt}{\large}

\titlespacing*{\chapter}{0pt}{-15pt}{10pt}
% \addcontentsline{lof}{chapter}{\protect\numberline{ \ref{exp1}}}
% \addcontentsline{lot}{chapter}{\protect\numberline{ \ref{exp1}}}
%************** End of Exp. Positioning *******


%%%%%%%%%%%%%%%%%%%%%%%%%%%%%%%%%%%%%%%%%%%%%%%
%%%%%%%%%%%%%%% Student's Part %%%%%%%%%%%%%%%%
%%%%%%%%%%%%%%% Start of Report %%%%%%%%%%%%%%%
%%%%%%%%%%%%%%%%%%%%%%%%%%%%%%%%%%%%%%%%%%%%%%%


%**********************************************
\chapter{Observation of Ratings of Various Electrical Machines}
\label{exp1}


%**********************************************
\section{Objectives}
The main objectives of this experiment are
\begin{enumerate}
  \item To understand the importance and application of the ratings on the machines.
  \item To memorize all the important ratings on the electrical machines in order to do experiments with them swiftly.
  \item To  notice the unit of each ratings of the machines.
\end{enumerate}

\section{Introduction}
\subsection{General ELectric Direct Current Generator}
The General Electric Direct Current Generator stands as a revolutionary device that has redefined the field of electrical power generation. Developed meticulously by the esteemed General Electric Company, this apparatus showcases cutting-edge advancements in electrical engineering. Its arrival marked a crucial turning point in the history of power supply, shifting from fragmented and inefficient sources to a more widespread and reliable electricity distribution system.
Through the implementation of state-of-the-art technologies, the General Electric Direct Current Generator effectively converts mechanical energy into a steady and precisely regulated direct current output. Its robust construction and precise engineering have cemented its position as a vital component in powering diverse industries, driving technological advancement, and providing illumination to cities worldwide. From its early inception to its continued relevance today, this extraordinary engineering achievement continues to shape the landscape of modern electrification, underscoring its significance as a symbol of human innovation.
\subsection{General Electric Synchronous Machine}
The General Electric Synchronous Machine is an exceptional representation of engineering excellence and has had a profound impact on the realm of electrical power generation. Created and meticulously crafted by the esteemed General Electric Company, this synchronous machine stands as a pinnacle achievement in the field of electrical engineering, shaping the trajectory of power generation and revolutionizing the methods of electricity harnessing and distribution.

Characterized by its cutting-edge design and advanced technology, the General Electric Synchronous Machine effectively converts mechanical energy into synchronous alternating current output. Its sturdy construction and precision engineering have established it as a fundamental component in various industrial applications, providing power to critical infrastructures and catalyzing technological advancement. From its inception to its continued relevance in contemporary power systems, this remarkable machine stands as a testament to human ingenuity and remains an integral element in the quest for dependable and sustainable electrical energy solutions.
\subsection{General Electric Induction Motor}
The General Electric Induction Motor represents a significant milestone in the realm of electrical engineering, exerting a profound impact on industrial applications and power generation. Developed with utmost precision by the esteemed General Electric Company, this motor marks a pivotal advancement, revolutionizing electromechanical conversion principles and serving as a fundamental pillar in contemporary electrical systems.

Distinguished by its inventive design and state-of-the-art technology, the General Electric Induction Motor adeptly transforms electrical energy into mechanical power, facilitating diverse industrial operations. Its sturdy build and meticulous engineering have firmly established its indispensability across various sectors, driving productivity, efficiency, and providing vital propulsion to essential machinery worldwide. Throughout its inception to its enduring relevance, this exceptional motor epitomizes human ingenuity, spearheading progress in electrical engineering and sustainable energy solutions.
%**********************************************
\section{Theory}
The nameplace rating of so,e electrical machines which are observed in labratory are given below:
\section{Table}
\begin{table}[hbt!]
  \centering

  \label{tab1}
  \begin{tabular}{|clcccl|}
    \hline
    \multicolumn{2}{|>{\centering\arraybackslash}p{3.5cm}|}{KW$4^{th}$} & \multicolumn{2}{c|}{VOL. TS 250}                  & \multicolumn{2}{c|}{AMP 18} \\ \hline
    \multicolumn{3}{|c|}{RPM 1450}                                      & \multicolumn{3}{c|}{WOUND COPM}                                                 \\ \hline
    \multicolumn{3}{|c|}{FLD AMPS 1.0 AS SH GEN}                        & \multicolumn{3}{c|}{FLD OHMS 25C 152.8}                                         \\ \hline
    \multicolumn{3}{|c|}{DUTY CONT 60 CRISE}                            & \multicolumn{3}{c|}{$E_{NCL}$ DP SERV FACT. 1.15}                               \\ \hline
    \multicolumn{3}{|c|}{SUIT AS SHP}                                   & \multicolumn{3}{c|}{1500/3000RPM 240V}                                          \\ \hline
    \multicolumn{3}{|c|}{MOD 5CD256G317}                                & \multicolumn{3}{c|}{SERXY1 - 1070}                                              \\ \hline
  \end{tabular}
  \caption{General ELectric\\DC Generator}
\end{table}
\begin{table}[hbt!]
  \centering

  \label{tab3}
  \begin{tabular}{|clcccl|}
    \hline
    \multicolumn{2}{|>{\centering\arraybackslash}p{2.5cm}|}{KW 2} & \multicolumn{2}{c|}{VOLTS 250/250}                & \multicolumn{2}{c|}{AMP 8.0} \\ \hline
    \multicolumn{3}{|c|}{RPM 1450}                                & \multicolumn{3}{c|}{WOUND COMP}                                                  \\ \hline
    \multicolumn{3}{|c|}{FLD AMPS 0.467}                          & \multicolumn{3}{c|}{FLD OHMS 25C 350}                                            \\ \hline
    \multicolumn{3}{|c|}{DUTY CONT 60 CRISE}                      & \multicolumn{3}{c|}{$E_{NCL}$ DP SERV FACT. 1.15}                                \\ \hline
    \multicolumn{3}{|c|}{MOD 50D218G192}                          & \multicolumn{3}{c|}{SER WYI-683}                                                 \\ \hline
  \end{tabular}
  \caption{General Electric \\ Direct Current Generator}
\end{table}
\begin{table}[hbt!]
  \centering

  \label{tab2}
  \begin{tabular}{|c|c|}
    \hline
    MODEL : 5SJ4254A2Y12  & SERIAL WY-7400B29                 \\ \hline
    TYPE: RPM 50 CYCLE    & 50 C RISE                         \\ \hline
    GENERATOR 4KVA 0.8 PF & 120/240 VOLTS 3PH 19.2/9.6 FL Amp \\ \hline
    MOTOR 6HP 1.0 PF      & 120/240 VOLTS 3PH 22/11 FL Amp    \\ \hline
    FIELD Amp MAX 2.1     & FIELD VOLTS 125                   \\ \hline
  \end{tabular}
  \caption{General Electric \\ Synchronous Machine}
\end{table}
\newpage
\begin{table}[hbt!]
  \centering

  \label{tab3}
  \begin{tabular}{|cc|}
    \hline
    \multicolumn{1}{|c|}{MODEL 2M4254A22Y26}          & SER NOXY 6454100                \\ \hline
    \multicolumn{1}{|c|}{HP3}                         & SERVICE FACTOR 1.15             \\ \hline
    \multicolumn{2}{|c|}{FL RPM 1420}                                                   \\ \hline
    \multicolumn{1}{|c|}{VOLTS 220/440}               & PHASE 3                         \\ \hline
    \multicolumn{1}{|c|}{FLAMPS 9.1/5.55}             & CYCLE 50                        \\ \hline
    \multicolumn{1}{|c|}{TYPE M}                      & FRAME 254V                      \\ \hline
    \multicolumn{1}{|c|}{C RISE 40}                   & TIME RATING CONT                \\ \hline
    \multicolumn{1}{|c|}{SEC VOLTS 145}               & SEC APMS 10.0                   \\ \hline
    \multicolumn{1}{|c|}{DRIVE END AF BMA BRG 40BC03} & 0PP DRIVE ENG AFBMA BRG B5 BC02 \\ \hline
  \end{tabular}
  \caption{General Electric\\
    Induction Motor}
\end{table}
\section{Conclusions and Discussions}
Two types of generator and two types of motor was seen in the lab. The generators are General electric direct current generator and general electric direct current generator. The motors are General electric synchronous machines and induction motor. The ratings was noted accurately, connection of the wires of the generator and the motor and the connection of the armature was seen. Digital images of each motors along with their ratings were taken to save the information for further experiments.
\newpage




%**********************************************




%%%%%%%%%%%%%%% Bibliography %%%%%%%%%%%%%%%%%%
%%%%%%%%%%%%%%% Uncomment only if you need %%%%
% If you need any citations then follow this:
% An article \cite{anarticle}\\
% A book \cite{abook}\\
% A series \cite{bookseries}\\
% Someone's thesis \cite{thesis}\\
% Some technical report \cite{report}\\
% A collection \cite{collection}\\
% Visited website \cite{website}\\
% Accepted for publication \cite{acceptedpub}\\
% Submitted for publication \cite{unpub}\\
% Not published \cite{notpub}\\
% Conversation \cite{conv}
% \addcontentsline{toc}{chapter}{\bibname}
% \bibliographystyle{IEEEtran} 
% \bibliography{asmbbiblio}
%************** End of Bibliography ***********



%**********************************************
%************** End of Student's Part *********
%************** End of the Report *************
%**********************************************
